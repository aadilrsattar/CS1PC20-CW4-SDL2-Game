% This file provides the generic header that sets up the page formatting and
% common custom macro and environment definitions that will be used for the
% C and C++ programming course.
%
% You can consult the `00-Example_Usage.tex` file for a full example of how to
% use this header file.
%

% The documentclass is set to an a4 paper article using KOMA-script
\documentclass[paper=a4]{scrartcl}
\PassOptionsToPackage{hyphens}{url}
\setlength\parindent{0pt}

% Hyperref is used, allowing for linking between different parts of the pdf.
\usepackage[breaklinks=true,colorlinks=true,linkcolor=red]{hyperref}
\hypersetup{urlcolor=red}
\def\UrlFont{\em}
\usepackage{attachfile}

% Geometry sets the exact dimensions of the page, overriding KOMA-scripts DIM.
\usepackage[top=0.5in,bottom=1in,left=0.5in,right=0.5in]{geometry}

% Tabu provides better tables than the built in tabular environment.
\usepackage{tabu}

% Allows us to use the H param for exact figure placement
\usepackage{float}
% Provides todos
\usepackage{todonotes}
\newcommand{\jk}[1]{\todo[inline,color=green]{JK: #1}}
\newcommand{\llb}[1]{\todo[inline,color=blue!20]{LLB: #1}}
\newcommand{\ob}[1]{\todo[inline,color=red!20]{OB: #1}}

% Adds additional section configuration options, these mirror `\date` and
% `\author` in behaviour.
\usepackage{totcount}
\makeatletter
\newcounter{cs@exercise}
\newcommand{\exercise}[1]{\setcounter{cs@exercise}{#1}}
\newcommand{\theexercise}{\arabic{cs@exercise}}
\newcounter{cs@xmarks}
\regtotcounter{cs@xmarks}
\newcommand{\totalxmarks}{\total{cs@xmarks}}
\newcommand{\addxmark}[1]{\addtocounter{cs@xmarks}{#1}}
\newcounter{cs@minutes}
\regtotcounter{cs@minutes}
\newcommand{\totalminutes}{\total{cs@minutes}}
\newcommand{\addminutes}[1]{\addtocounter{cs@minutes}{#1}}
\newtoks{\internal@term}
\newcommand{\term}[1]{\internal@term={#1}}
\newcommand{\theterm}{\the\internal@term}
\makeatother

% The following are used to configure the footer of the document.
\usepackage{lastpage}
\usepackage{fancyhdr}
\usepackage{titling}
\usepackage{textcomp}
\pagestyle{fancy}
\fancyhf{}
\rhead{}
\lhead{}
\lfoot{CS1PR, Tutorial -- Week \theexercise{}}
\cfoot{\thedate}
\rfoot{\thepage/\pageref{LastPage}}

% Listings is used for the inclusion of source code.
\usepackage{xcolor}
\usepackage{textcomp} %for upquote=true (straight quotes)
\usepackage{listings}
\definecolor{pblue}{rgb}{0.13,0.13,1}
\definecolor{pgreen}{rgb}{0,0.5,0}
\definecolor{pred}{rgb}{0.9,0,0}
\definecolor{pgrey}{rgb}{0.46,0.45,0.48}
\lstset{
	basicstyle=\ttfamily\footnotesize,
	numberstyle=\ttfamily\tiny,
	frame=single,
	numbers=left,
	numbersep=1em,
	xleftmargin=2em,
	language=[AlLaTeX]TeX,
	breaklines=true,
	breakatwhitespace=true,
	postbreak=\hbox{$\hookrightarrow$ },
	showstringspaces=false,
	tabsize=2,
    commentstyle=\color{pgreen},
    keywordstyle=\color{pblue},
    texcsstyle=*\color{pblue},
	stringstyle=\color{pred},
	moredelim=[il][\textcolor{pgrey}]{\[\]},
	moredelim=[is][\textcolor{pgrey}]{\%\%}{\%\%},
    columns=fullflexible,
    upquote=true,
    morekeywords={maketitle,subsection,article}
}
\usepackage{cleveref}

\parskip 8pt

\term{Autumn 2019}

% The following defines the command \makesheetheader which is used to create
% the header shown at the top of every exercise sheet.
\definecolor{infobox}{HTML}{e0f7fa}
\newcommand{\makesheetheader}{%
    \thispagestyle{empty}
    \begin{center}
    	\begin{tabular*}{\textwidth}{@{}l@{\extracolsep{\fill}}r@{}}
    	  University of Reading & \LaTeX Tutorial \\
    		Department of Computer Science	& CS1PR / \theterm \\
         & \totalminutes{} Minutes\\
    		\theauthor & Date: \thedate \\
    		& \\
    		\hline
    	\end{tabular*}
    \end{center}%
}

% The following commands are used to declare the starts of each section, with how long they are expected to take and how many marks are expected if any.
\newcommand{\generaltoc}[2]{%
	\addtocounter{section}{1}%
	\section*{\arabic{section}. #2: #1}%
	\addcontentsline{toc}{section}{\arabic{section}. #2: #1}%
	\setcounter{subsection}{0}
}
\newcommand{\introduction}[1]{%
		\generaltoc{#1}{Introduction}
}

\newcommand{\tutorial}[2]{%
		\generaltoc{#1}{Tutorial (#2 Minutes)}%
		\addminutes{#2}%
}
\newcommand{\groupwork}[2]{%
    \generaltoc{#1}{Group Work (#2 Minutes)}%
    \addminutes{#2}%
}

% The following define reusable subsections.
\newenvironment{critera}{%
    \subsection*{Marking Criteria}
    \begin{itemize}
}{%
    \end{itemize}
}
\newenvironment{criteria*}{
    \subsection*{Marking Criteria}
}{}
\newenvironment{steps}{%
    \subsection*{Steps}
    \begin{enumerate}
}{%
    \end{enumerate}
}
\newenvironment{steps*}{
    \subsection*{Steps}
}{}
\newenvironment{literature}{%
    \subsection*{Further Reading}
    \begin{itemize}
}{
    \end{itemize}
}
\newenvironment{literature*}{
    \subsection*{Further Reading}
}{}
\newenvironment{hints}{%
    \subsection*{Hints}
    \begin{itemize}
}{
    \end{itemize}
}
\newenvironment{hints*}{
    \subsection*{Hints}
}{}
% These are not currently used, and if the previous definitions were anything
% to go by should likely be replaced by environments also.
% \newcommand{\Input}{\subsection*{Input}}
% \newcommand{\Output}{\subsection*{Output}}
% \newcommand{\example}{\subsection*{Example}}

\newcommand{\cmd}[1]{\texttt{\$~#1}}
\newcommand{\func}[1]{\texttt{#1()}}

\newcommand{\includeLaTeX}[1]{%
    Source Code: File \texttt{#1.tex} (Code: \attachfile{#1.tex}) (PDF: \attachfile{#1.pdf}) \\%
    \lstinputlisting{#1.tex}%
}
